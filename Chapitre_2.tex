\chapter{Calcul préliminaire}
\section{Calcul préliminaire des $P_i^{sat}$: Paramètres d'Antoine }
Pour estimer les pressions de vapeur des liquides ou des solides pu,les chimistes utilisent souvent l'équation de Clausius-Clapeyron. Plusieurs des hypothèses retenues pour le calcul de l'équation échouent à haute pression et à proximité du point critique, et dans ces conditions, l'équation de Clausius-Clapeyron va donner des résultats inexacts.

Les ingénieurs chimistes, ayant besoin d'estimations plus fiables de pression de vapeur, utilisent l'équation d'Antoine; une simple équation à 3 paramètres de la forme:\\

\begin{equation}
Log_{10}( P^{sat} ) = A + \cfrac{B}{T+C} \\
\end{equation}

où A, B, et C sont des coefficients d'Antoine qui varient d'une substance à une autre.
\subsection{Cas de l'Acétone}

